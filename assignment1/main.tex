\documentclass[11pt]{scrartcl}
\usepackage[utf8]{inputenc}
\usepackage{graphicx,wrapfig,float,amsmath,multicol}
\usepackage[section]{placeins}
\usepackage{pgf,tikz,pgfplots}
\usepackage{pgfplots}
\usepackage{hyperref}
\hypersetup{
    colorlinks=true,    % Colored links
    urlcolor=blue,       % Color of URLs
    linkcolor = blue
}
\usetikzlibrary{calc}
\usepgfplotslibrary{groupplots}
\pgfplotsset{compat=1.15}
\usepackage{mathrsfs}
\usetikzlibrary{arrows}
\usepackage[english]{babel}
\usepackage{a}
\usepackage{graphicx}
\usepackage[sexy]{evan}
\usepackage[dvipsnames]{xcolor}
\author{ by Mani Sarthak \& Harshit Gupta}
\date{\today}
\setlength{\parindent}{0pt}
\usepackage{caption}
\usepackage{amssymb}
\captionsetup[table]{labelsep=space}
\setcounter{tocdepth}{5}
\graphicspath{ {./images/} }


%COL 216 here
\title{Computer Networks}
\subtitle{Assignment1}



\begin{document}

\maketitle

Some points to be considered:
\begin{itemize}
    % \item Changing associativity and increasing cache size do affect the latency of the cache but here we are not considering such effects. As expected this will result in slightly different results as compared to the actual scenario.
    % \item All the plots are generated from the Python matplotlib library, the code for which is provided \href{https://drive.google.com/drive/folders/1ggkQp2t3NUP2j-lU_Dst7DpPSrZ3AbAS?usp=sharing}{\textcolor{blue}{here}} (in the file named run.py see createPlot).
    % \item The summary containing the data of time taken (in ns) is \href{https://drive.google.com/file/d/1tbcX3iz1-sOLc5bymLP_e-yGbUJjtBvu/view?usp=sharing}{\textcolor{blue}{here}}.
    % \item The plots used in the report can be seen \href{https://drive.google.com/drive/folders/170nIbyIFCMwzpfevd__3WixwtmWZnnlW?usp=share_link}{\textcolor{blue}{here}}.
    
    \item All the codes are managed on GitHub and can also be found in this \href{https://drive.google.com/drive/folders/1QKKCJGQ5x3ZLXGFgR8Z5mXXcyV3pBJWg?usp=sharing}{\underline{folder}}. 
    \item We both have Macbook so any functionality that is only specific to Windows Command Shell or Power Shell couldn't be concluded by us.
    

    
    
\end{itemize}

\vfill
\begin{center}
   Token Distribution:
\\
1. Mani Sarthak : 2021CS10095 : 10
\\
2. Harshit Gupta \ : 2021CS10552 : 10 
\end{center}




\newpage
\section{ Part1 : Network Analysis}
\subsection{Traceroute to servers}
Procedure: Connected personal laptop (Mani's Macbook) to 4G/5G cellular network of Jio using cellular hotspot. Then traceroute to different servers were called.
\subsubsection{Tracerouting to \underline{www.iitd.ac.in}}
\includegraphics[width = 15 cm, height = 18 cm]{images/iitd.ac.in.png}
It seems as the ISP is blocking packets to the path to \underline{www.iitd.ac.in}, so we now try to traceroute to \underline{www.google.com}.
\subsubsection{Tracerouting to \underline{www.google.com}}
\includegraphics[width=15 cm, height = 8 cm]{images/google.com.png}
The traceroute to \underline{www.google.com} was successful and it took 9 hops to reach the destination address.
\subsection{Observations}
\begin{itemize}
    \item After calling traceroute to multiple destinations, i didn't find any case in which paths default to IPv6, so i conclude that if that occurs then it must be rare.
    \item By default in Mac's unix running traceroute returns IP addresses in paths in IPv4 only, however if one wants to get the IP's in IPv6 then one should use the command traceroute6.
    \item We can also observe private IP addresses in the paths like 
    \underline{192.168.249.240}, \underline{192.168.31.1}, \underline{198.162.13.38} and  so on and these are common in the path of both. This shows that they both use the initial routers of Jio Network which is private before transferring the packets to the global networks.
    \item There are also routers that do not reply to traceroute requests. Example: see router number 7 in google.com traceroute call.
\end{itemize}

\subsection{Ping}
\begin{itemize}
    \item Ping command is used to send packets to check wether an IP address is live or not.
    \item The -s flag allows us to send ping requests with different packet sizes with the default value being 56 Bytes (excluding 8 Bytes ICMP header). The real packet which is sent and received is 8 Bytes more than the data pinged, this is because of the 8 Byte ICMP header which is added to the original data.
    \item The maximum size of ping packets that we are able to send from MacBook M1 by using IITD WiFi for websites is mentioned below. We have used \underline{Binary Search} to get to the conclusion.
\end{itemize}
\begin{table}[ht]
\centering
\begin{tabular}{|c||c|}
\hline
 Destination & Size  \\
 \hline
\hline
www.facebook.com & 1472 \\
\hline
www.nytimes.com & 1472\\
\hline
www.indianexpress.com &  8184\\
\hline
www.iitd.ac.in & 8184 \\
\hline
\end{tabular}
\caption{Table showing the maximum size of ping packets without ICMP header (in Bytes) that can be sent to the destination servers.}
\label{tab:simple}
\end{table}


\newpage
Observations:
\begin{itemize}
    \item  As servers like www.iitd.ac.in and www.indianexpress.com have very higher sizes. Also because www.facebook.com have multiple servers and the one we are sending ping is also in proximity to us but still we are getting lesser size, can be because packet size for ping also depends on servers. 
    \item So we conclude that the maximum size depends on the servers that we are sending and also that whether they are in our proximity or not.
\end{itemize}



\newpage
\section{ Part2}
\subsection{Methodology}
\textbf{Logic} : what we did here was very simple. we used linux ping command here. we observed that when we limit the TTL ( max no. of hopes) such that it can't reach the destination IP, the router sends a ICMP response upon dropping the packet which contain the IP address at which packet is dropped. so by using this observation , we iterate and increased the TTL at each iteration so that we get all intermediate IP address as no. of hops is increased by 1 each time until the packet is reached to the destination and we stop there. after that we ping at each intermediate IP to get \textbf{Round Tip Time} and we report it. so traceroute functionality is replictaed.

\subsection{View of code}
\includegraphics[width = 17 cm , height = 15 cm]{Screenshot 2023-08-12 at 4.45.44 PM.png}
\subsection{Commands and Modules used}
\begin{itemize}
    \item ping -c 1 -m \textit{TTL} \textit{Destination\_IP} here -c is for sending only 1 packet and -m is for defining TTL ( here macos terminal is used)
    \item ping -c 1 \textit{destination\_ip} for getting the time for each intermediate IP 
    \item \textbf{subprocess} module is used for running commands at console
    \item \textbf{re} module is used for extracting IP from text output
    \item \textbf{sys} module for taking input from command line 
    \item \textbf{socket} module for getting ip address for destination website
\end{itemize}
\subsection{Input}
use the code and put input in console as:-\\ \\
python/python3 script\_trace.py \textit{destination\_ip}
\subsection{Output for \underline{www.iitd.ac.in}}
\includegraphics[width = 17 cm, height = 4 cm]{Screenshot 2023-08-13 at 4.32.25 PM.png}

\subsection{design decisions}
\begin{itemize}
    \item If in a particular IP we can't get a response or we can't get a intermediate IP during hops we leave a it blank with a \textbf{*} mark
    \item  we are sending 3 packet here so Round Trip Time here is for 3 packets only and we can't change no. of packets as it is by default 3 in tarceroute as sending more than 1 packet will make many edge cases
\end{itemize}

\newpage
\section{ Part3 }
\subsection{Hops analysis}

For India IITD WiFi is used, for South Africa the traceroute server of Capetown city was used and for USA traceroute server of Seattle was used.
\begin{table}[ht]
\centering
\begin{tabular}{|c|c|c|c|c|c|}
\hline
Source \& Destination & utah.edu & uct.ac.za & iitd.ac.in & google.com & facebook.com \\
\hline
India & 64+ & 64+ & 4 & 11 & 13 \\
\hline
South Africa & 19 & 4 &  20 & 13 & 19\\
\hline
USA &  11 & 15 & 16 & 10 & 9\\
\hline
\end{tabular}
\caption{Table showing the number of hops to the destination from different traceroute servers.}
\label{tab:simple}
\end{table}

Observations:
\begin{itemize}
    \item If the traceroute source and destination are close to each other geographically then they have roughly fewer hops between them. For example see (India, iitd.ac.in) , (South Africa, uct.ac.za), (USA, utah.ac.in).
    \item Google and Facebook have very  few number of hops as compared to others. This is because they have distributed servers across the globe and the traceroute takes their closest server as destination.
\end{itemize}

\subsection{Latency Analysis}
For India IITD WiFi is used, for South Africa the traceroute server of Capetown city was used and for USA traceroute server of Seattle was used.
\begin{table}[ht]
\centering
\begin{tabular}{|c|c|c|c|c|c|}
\hline
Source \& Destination & utah.edu & uct.ac.za & iitd.ac.in & google.com & facebook.com \\
\hline
India & NA & NA & 5.066 & 8.023 & 31.093 \\
\hline
South Africa & 231.681 & 4.013 &  320.114 & 248.818 & 228.993\\
\hline
USA &  22.038 & 294.328 & 270.981 & 74.357 & 86.714\\
\hline
\end{tabular}
\caption{Table showing the latency between destinations and different traceroute servers (in ms).}
\label{tab:simple}
\end{table}

Observations:
\begin{itemize}
    \item Latency seems to be related to the number of hops. Latency depends on a number of factors like traffic between the routers, processing time at the routers etc. Also higher number of hops means a higher chance of getting stuck in traffic and more time consumed at routers. Hence increase in number of hops increases the latency as well.

\end{itemize}


\subsection{Distributed servers}
We find that servers like \underline{www.iitd.ac.in}, \underline{www.utah.edu} and \underline{www.uct.ac.za} are resolved to same IP address irrespective of from where they are getting the traceroute request, whereas servers like \underline{www.google.com} and \underline{www.facebook.com} have different destination IP's for different parts of world. 
\\
The reason for this is because big MNC's like google and Facebook need to provide their users with as low latency as possible and also that is economically optimal for them so they create their databases across the globe each acting as a server with unique IP. However educational institutes like \underline{utha.edu}, \underline{iitd.ac.in} have a single server for them. 
\subsection{Distributed Servers of Google and Facebook}
The IPv4 addresses of Google and Facebook are obtained by sending  traceroute requests to them through different locations namely:
\begin{itemize}
    \item IITD WiFi.
    \item Princeton traceroute server \underline{www.net.princeton.edu}.
    \item HanNet Germany server.
\end{itemize}
The IPv4 addresses of the destination is summarised in the table below.

\begin{table}[ht]
\centering
\begin{tabular}{|c|c|c|}
\hline
Source \& Destination & www.google.com & www.facebook.com \\
\hline
IITD Wifi & 142.250.206.142 & 157.240.16.35  \\
\hline
Princeton & 142.250.65.238 & 31.13.71.36 \\
\hline
HanNet &  172.217.18.14 & 157.240.210.35\\
\hline
\end{tabular}
\caption{Table showing the IPv4 address of various Google and Facebook servers.}
\label{tab:simple}
\end{table}


The number of hops and latency for the traceroute requests between the source and destinations can be summarised in the tables below (for latency average of the latency by three packets sent is taken).


\begin{table}[ht]
\centering
\begin{tabular}{|c|c|c|}
\hline
Source \& Destination & www.google.com & www.facebook.com \\
\hline
IITD Wifi & 11 & 13  \\
\hline
Princeton & 20 & 20 \\
\hline
HanNet &  23 & 24\\
\hline
\end{tabular}
\caption{Table showing \textit{\#}hops to various Google and Facebook servers from IITD WiFi.}
\label{tab:simple}
\end{table}




\begin{table}[ht]
\centering
\begin{tabular}{|c|c|c|}
\hline
Source \& Destination & www.google.com & www.facebook.com \\
\hline
IITD Wifi & 8.373 & 38.891  \\
\hline
Princeton & 334.253 & 314.255 \\
\hline
HanNet &  404.218 & 201.761\\
\hline
\end{tabular}
\caption{Table showing the latency of various Google and Facebook servers from IITD WiFi.}
\label{tab:simple}
\end{table}


Observation:
\begin{itemize}
    \item Here also we see that traceroute requests to same server but with different IP address have different paths and latencies.
    \item Also the latencies and \textit{\#}hops are related to each other as increase in \textit{\#}hops roughly increases the latency too.
    \item Last but not the least, IP addresses for the server nearer to the source have very low latency as compared to other IP addresses.
    \item Some IP adresses are optimised for my location and hence they have very few hops and latencies but other IP's that are way too far off and hence are longer.
\end{itemize}


\subsection{Tracerouting to Google and Facebook from different sources}

\begin{table}[ht]
\centering
\begin{tabular}{|c|c|c|}
\hline
Source \& Destination & www.google.com & www.facebook.com \\
\hline
Los Angeles, USA & 8 & 9  \\
\hline
Berlin, Germany & 14 & 12 \\
\hline
London, UK &  12 & 12    \\
\hline
Melbourne, Australia & 7 & 1 \\
\hline
Johannesburg, South Africa & - & 10 \\
\hline
Mombasa, Kenya & 17 & 15 \\
\hline
\end{tabular}
\caption{Table showing the number of hops to destinations from different traceroute servers.}
\label{tab:simple}
\end{table}




\begin{table}[ht]
\centering
\begin{tabular}{|c|c|c|}
\hline
Source \& Destination & www.google.com & www.facebook.com \\
\hline
Los Angeles, USA & 7.861 & 10.717  \\
\hline
Berlin, Germany & 150.234 & 157.013 \\
\hline
London, UK &  131.682 & 128.631\\
\hline
Melbourne, Australia & 11.528 & 11.823 \\
\hline
Johannesburg, South Africa & - & 226.6 \\
\hline
Mombasa, Kenya & 340.216 & 310.067 \\
\hline
\end{tabular}
\caption{Table showing the average latency to destinations from different traceroute servers (in ms)}
\label{tab:simple}
\end{table}


Observations:
\begin{itemize}
    \item We notice that the \textit{\#}hops and latency for South African cities are significantly more than that of cities like Los Angeles, Melbourne etc.
    \item Assuming that the most developed cities like Los Angeles, Melbourne etc must have their local ISP's directly peered with Google and Facebook. We can say that cities like Mombasa and Johannesburg doesn't have their local ISP's directly peered with Google and Facebook.
\end{itemize}

\newpage
\section{Part4}
\subsection{Part a}
In this part we started wireshark and captured the packets for 2 sites \underline{www.iitd.ac.in} and \underline{http://act4d.iitd.ac.in}.\\ \\
Now we apply the DNS filter and got all the dns requests and responses and after subtracting the response time with request time we get the time for dns request-response, for www.iitd.ac.in it is:



\begin{table}[ht]
\centering
\begin{tabular}{|c|c|c|}
\hline
Response time \& type of request & www.iitd.ac.in & http://act4d.iitd.ac.in \\
\hline
AAAA & 7.694 ms & 3.916 ms  \\
\hline
A & 7.647 ms & 3.853 ms \\
\hline
HTTP &  7.602 ms & 3.798 ms\\
\hline
\end{tabular}
\caption{Table showing the response time with different request for 2 sites (in ms)}
\label{tab:simple}
\end{table}




\includegraphics[width = 17.5 cm, height = 2 cm]{Screenshot 2023-08-13 at 4.35.39 PM.png}
pic for www.iitd.ac.in\\ \\
\includegraphics[width = 17.5 cm , height = 2 cm]{Screenshot 2023-08-13 at 4.37.15 PM.png}
pic for http://act4d.iitd.ac.in
% \caption{ pic for www.iitd.ac.in}
\subsection{Part b}
Now applying the http filter and get those packets 
\begin{itemize}
    \item no. of http request generated for www.iitd.ac.in is :- \textbf{1} beacuse this site has secured http so 1 request is visible not css ,js etc.
    \item no. of http request generated for http://act4d.iitd.ac.in is :- \textbf{12} beacuse this website is not secured  so we can see how many request these pages have
\end{itemize}
with seeing the http request for act4d website we can see how webpages are structured and how browser render complex images and files. we can see. so we can see that first application javascript is requested and then templates pages such as html , css , text etc. are requested and then rendered. after they are completed media pages such as images, video , favicon etc. are requested and then rendered.

\includegraphics[width = 17.5 cm, height = 3 cm]{Screenshot 2023-08-13 at 4.36.23 PM.png}
pic for www.iitd.ac.in\\ \\
\includegraphics[width = 17.5 cm , height = 10 cm]{Screenshot 2023-08-13 at 4.38.08 PM.png}
pic for http://act4d.iitd.ac.in

\subsection{Part c}
Now we apply the ((ip.src==192.168.1.3 \&\& ip.dst==10.7.174.111) or (ip.src==10.7.174.111 \&\& ip.dst==192.168.1.3)) \&\& TCP filter and answer of 3 questions are given below:
\begin{itemize}
    \item now we need to find the no. of tcp connections opened between my browser and web-server. as stated a tcp connection is 3-way handshake, our browser,client, sends SYN message to server , server replies with SYN-ACK message and then client sends an ACK. now to find no. of connecteions opened we just need to count no. of SYN request and we get that count as \textbf{6}
    \item now we find that this is not same and less than number of HTTP requests for content objects in prevoius part due to the fact that multiple http objects get transferred over same tcp connection due to HTTP/1.1 protocol being alive to reduce latency
    \item Yes, as stated above multiple HTTP content being fetched over same TCP connection due to the fact the HTTP/1.1 protocol is present and its ability to reuse existing connections for multiple requests to reduce latency.
\end{itemize}
\subsection{Part d}
Now applying trace for \textbf{http://www.indianexpress.com} and we apply the http filter. suprisingly we found that there is no HTTP traffic visible over the network as we only found 1 http request for indianexpress.com visible of HTTP/1.1 protocol visible \\ \\
now bowsing through the entire trace without any filters but we cannot see any contents of html and javascript files being transfered. so the reason for that is that this website and many other uses \textbf{HTTP secure}  which encrypts the traffic, making it difficult to directly see the contents of the transferred files in plaintext. \\ \\
therefore we are not able to view the HTTP traffic like the act4d.iitd.ac.in 
\newpage

\newpage





\end{document}
